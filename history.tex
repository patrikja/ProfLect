\documentclass[11pt,twoside,twocolumn]{article}
\usepackage{bibentry}
\usepackage{wrapfig}
\usepackage[numbers,sort&compress]{natbib}
\usepackage[a4paper,margin=2.5cm]{geometry}
\usepackage{pdfpages}
\usepackage[utf8]{inputenc}
\usepackage{hyperref}
\usepackage{url}
\usepackage{parskip}
\newenvironment{timelist}
{\begin{description}
 \addtolength{\itemindent}{1em}
 \setlength{\itemsep}{0cm plus 0.1cm minus 0.1cm}}
{\end{description}}
\newcommand{\todo}[2][?]{\marginpar{\raggedright \tiny \textbf{TODO (by #1):} #2}}

\providecommand{\cvpart}[1]{\subsubsection*{#1:}}
\providecommand{\cpp}{C\kern-0.05em\texttt{+\kern-0.03em+}}
\providecommand{\Cpp}{\cpp}
\usepackage[english]{babel}
\usepackage{uri}

\let\oldmarginpar\marginpar
\renewcommand\marginpar[1]{\-\oldmarginpar[\raggedleft\footnotesize #1]%
{\raggedright\footnotesize #1}}
\providecommand{\TODO}[1]{\marginpar{#1}}

\usepackage{tocloft}
\providecommand{\noc}[1]{\newline Number of citations: \textbf{#1}.}
\title{Coherent description of my collected research work}
\author{Patrik Jansson}
\date{2016-05-09}
\begin{document}
\maketitle

\section{PhD studies}
I started my research career as a MSc thesis student in the fall of
1994 with Johan Jeuring as my supervisor.
%
He had coined the term ``Polytypic programming'' for what I would
nowadays call datatype generic programming and as I became a PhD
student we set out to chart the territory where category theory and
initial algebras meet programming.
%
I immediately joined in writing the lecture notes for the summer
school on Advanced Functional Programming \citep{jeuringjansson-afp}
and I developed an implementation of these ideas as a language
extension for the functional programming language Haskell~\citep{janssonjeuring1997a}.
%
When I defended my licentiate thesis \citep{jansson97a} in 1997 my supervisor
left Sweden---first for the software company Baan Inc.\ and later for a
permanent position at Utrecht Univ.
%
I still had some supervision until my PhD, but it was at a distance.
%
This meant that already from the third year of my PhD position I had
to lead my own research projects.
%
(I have continued to collaborate with Jeuring now and then.)


With the base theory and the implementation in place I set out to
implement a library of polytypic functions
\citep{janssonjeuring-polylib} and I returned to and further developed
my MSc thesis topic into a journal publication
\citep{janssonjeuring-polyunify}.
%
I made a few longer research visits: to Yale Univ.\ and to Univ.\ of Oxford
 where I worked on lecture notes on Generic Programming
\citep{backhouseetal98}, the underlying theory and implementation
\citep{jansson:PolyP2compiler} and applications to data conversion
\citep{janssonjeuringdc} and rewriting
\citep{janssonjeuringWGP00:rewriting}.
%
In 1999--2000 I was part time on parental leave and part time
finishing my PhD thesis as a monograph \citep{jansson-phdthesis}, combining
and unifying the earlier results.

\section{Assistant Professor}
After my PhD I broadened my competence by teaching new courses and I
took a leading role in the department administration (Director of
Studies, member of the steering group, later Vice Head of Department).
%
I also developed the results on Arrows and Data Conversion into a
journal paper \citep{janssonjeuring-dataconv} and I applied for, and
was granted, a VR project to work on \emph{Generic Functional Programs
  and Proofs} (1.8M SEK, 2003--2005).
%
In this project I worked with a PostDoc (Marcin Benke) with whom I
wrote \citep{benkedybjerjansson2003:gendepty} but I also worked with my
first PhD student Ulf Norell (PhD 2007).
%
We continued development of PolyP (from my PhD thesis) resulting in a
Haskell library version \citep{jansson:PolyP2compiler} and two more
papers \citep{norelljansson2003:PolyHaskell, norelljansson04:ProtoGen}.
%
I identified Dependent Types as a natural setting for generic
programming, and Norell moved on to work on developing the dependently
typed language Agda together with my second PhD student Danielsson
(also PhD 2007).
%
With Danielsson I worked on program correctness through types
resulting in two co-authored papers
\citep{danielssonjansson04:chasingbottoms,
  danielssonetal06:fastandloose} and a few more with only him as
author.
%
I am proud to say that all my graduated PhD students (Norell,
Danielsson and later Bernardy) are still successful in academia, now
themselves at the assistant professor level.

\subsection{Cover project}
In 2003 I was also co-applicant on a project called \emph{Cover ---
  Combining Verification Methods in Software Development} which was
funded with 8M SEK by the Swedish Foundation for Strategic Research.
%
Within this project I worked hard to integrate applications, software
libraries, and theories from the Programming Logic group (mainly Agda
and its standard library) and the Multi group (mainly QuickCheck and
automatic first order theorem provers).
%
I was active in writing the grant application and I also took an
active part in organisation.

\section{Associate Professor}
After the Cover project was finished, I returned to and broadened my
work on generic programming into three tracks (Haskell, \Cpp, Agda):
%
\paragraph{Haskell} I wrote two papers with the generic programming
group in Utrecht: one on testing \citep{janssonjeuring2007:Testing} and
one review paper comparing many generic programming libraries in Haskell:
\citep{Rodriguezetal2008:CompGenLib}.
%
I also started working with J-P. Bernardy as my third PhD student.


\paragraph{\Cpp} I joined the Software Methodologies and Systems (SMS)
group at Chalmers and led a project to understand Generic Programming
(in the \Cpp{} sense) in depth by working out and carefully exploring the
correspondence between Haskell type classes and \Cpp{} Concepts.
%
This resulted in new results, two papers \citep{SMS-Bernardy:2008:CCC,
  bernardy_generic_2010}, a joint workshop \citep{wgp09proceedings} and
new collaborations but the local SMS group dissolved when Prof. Schupp got a
professors' chair in Germany.

\paragraph{Agda} I initiated work on the Algebra of Programming in
Agda (AoPA) and wrote two papers together with Mu and Ko
\citep{mukojansson08:mpc:dcc, MuKoJansson2009AoPA}.
%
I also developed and taught a PhD level course on Algebra of
Programming twice.
%
AoPA is using and building further on results by me, Norell and
Danielsson.
%

\subsection{Climate Impact Research}
A new interesting development started in 2007 when I got involved with
the ``Potsdam Institute for Climate Impact Research (PIK)''.
%
They want to develop correct software implementing models in use for
simulating global systems (both for economy and climate) and they
searched for a partner knowing the Algebra of Programming.
%
We have established a collaboration over the years (leading to
publications, EU grants and even an EU call) which I will get back to
later in Section \ref{sec:GSS}.


\section{Professor}

In 2011 I was promoted to Professor in Computer Science at Chalmers
and from 2011--2013 I was Head of the 5-year programme in Computer
Science and Engineering (CSE).
%
In 2013 I was appointed Head of the Division of Software Technology
where I lead around 50 employees.

\subsection{Programs and Proofs}

I was main applicant and project leader for \emph{Strongly Typed
  Libraries for Programs and Proofs} funded with \textbf{2.4M SEK} by
VR (2011--2015).
%
I worked with J-P. Bernardy (AssProf) and C. Ionescu (PostDoc) on
correctness of reusable software components.
%
This project is at the core of my research work and my activities in the
other parallel projects are all somewhat overlapping (I make sure to
formulate workpackages and subprojects to fit with my core interests).

\subsection{Parametricity}
With Bernardy I have been exploring parametricity theory and we have
published one paper on polymorphic testing
\citep{bernardy_testing_2010} and some very interesting results on
Parametricity for Dependent Types \citep{bernardy_parametricity_2010,
  bernardy_proofs_2012}.
%
My feeling is that the results of this project are among the best I
have ever been involved in.

\subsection{Domain Specific Languages}

I was co-applicant on ``\emph{Software Design and Verification using
  Domain Specific Languages}'' funded with \textbf{11M SEK} by the
Swedish Science Council (VR, multi-project grant in strategic
Information and Communication Technology).
%
I was active in the steering group, recruiting PhD students, PostDocs
and an Assistant Professor.
%
Research-wise I worked with my most recent PhD student (Jonas
Duregård) on theory and applications of Domain Specific Languages
leading to publications on Embedded Parser Generators
\citep{BNFC-meta-Haskell2011} and Functional Enumeration
\citep{duregardHaskell12Feat}.

I am also co-applicant for ``RAW FP: Productivity and Performance
through Resource Aware Functional Programming'' granted with \textbf{25M SEK}
by SSF (2011--2016).
%
I have been active in recruitment, organisation and research (mainly
the work package ``DSL framework'').
%
In addition to the publications mentioned above we also published a
paper on Class Laws in Haskell \citep{jeuringHaskell12ClassLaws}
%
and one on ``Certified Context-Free Parsing: A formalisation of
Valiant's Algorithm in Agda'' \citep{bernardy_certified_2015}.

My most recent project on DSLs is
``\href{http://wiki.portal.chalmers.se/cse/pmwiki.php/FP/DSLsofMath}{Domain-Specific
  Languages of Mathematics}'', supported by Chalmers Education Quality
Funding (2014--2016).
%
This project has lead to one new BSc level course, the project ideas
are described in the pedagogical paper
\citep{TFPIE15_DSLsofMath_IonescuJansson},
%
and ``... ideally, the course would improve the mathematical education
of computer scientists and the computer science education of
mathematicians.''


\subsection{Global Systems Science}\label{sec:GSS}

In collaboration with PIK I have written three papers, each by
combining knowledge from two fields: one combining \Cpp{} and Haskell
\citep{LinckeJanssonetalDSLWC2009}, one combining testing and proving
\citep{ionescujansson:LIPIcs:2013:3899} and one using dependent types
for scientific computing~\citep{ionescujansson2013DTPinSciComp}.
%
I'm very grateful to C. Ionescu who has been my main contact trough
all these years.

We also managed to secure European funding from the the FP7 FET-Open
scheme to develop a research programme for ``Global Systems Dynamics
and Policy'' (1.3M EUR, 2010--2013).
%
As work package leader and site leader I organised several workshops,
wrote deliverables and participated in meetings with researchers and
practitioners from other fields.

The explicit aim of this work was to establish ``Global Systems
Science (GSS)'' as a multi-disciplinary research area and to secure further
funding from the EU commission.
%
The call
\textbf{\href{http://ec.europa.eu/research/participants/portal/desktop/en/opportunities/h2020/topics/2074-fetproact-1-2014.html}{FETPROACT1}}
(Future and Emerging Technology, Proactive support for GSS) in Horizon
2020 is concrete evidence on the success of this line of work.
%
I am now site- and workpackage leader in the project GRACeFUL:
``Global systems Rapid Assessment tools through Constraint FUnctional
Languages'' granted (with \textbf{2.4M EUR}, 2015--2018) by the
FETPROACT1 call.
%
In the GRACeFUL project I will work with my new PhD student (I. Lobo
Valbuena) who started 2015-08-01.

From 2015-10-01 I'm also site-leader for Chalmers in a ``Center of
Excellence for Global Systems Science'' (CoEGSS, 2015--2018).
%


\section{Best results}
Among my best results I count
\begin{itemize}
\item my PhD graduates: Norell, Danielsson, and Bernardy
\item Parametricity and dependent types \citep{bernardy_parametricity_2010}
\item Algebra of Programming in Agda \citep{MuKoJansson2009AoPA}
\item Fast and Loose Reasoning \citep{danielssonetal06:fastandloose}
\item Polytypic Data Conversion Programs \citep{janssonjeuring-dataconv}
\item Feat: functional enumeration of algebraic types \citep{duregardHaskell12Feat}
\item early work on Generic Programming \citep{backhouseetal98, janssonjeuring1997a} (well cited)
\item the Global Systems Science work \cite{jaeger13:GSSshort} leading to the FETPROACT1 call
\item establishing the Bologna structure (3y BSc + 2y MSc) at the CSE
  department at Chalmers (in my role as Vice Head of Department)
\item the self-evaluation reports (for the CSE degrees) as part of the
  national evaluation of all degrees (in my role as Head of the CSE
  programme). The BSc level even got the grade ``very high quality''.
\end{itemize}

\newpage
\section{Future plans}

I will
\begin{itemize}
\item lead the development of the education and research at the
  department in my role as head of the Software Technology division

\item take part in education at the BSc, MSc and PhD level (as
  teacher and examiner)

\item continue as supervisor (currently of J. Duregård and I. Lobo
  Valbuena) and examiner (of D. Rosén, A. Ekblad and M. Aronsson) in
  the CSE PhD school

\item lead and perform research, both in my own group and as part of
  larger consortia

\item keep applying for external research funding (currently two
  in the pipeline)

\item take part in exchange and communication with the surrounding
  society, mainly in the Global Systems Science (GSS) area

\item support more junior colleagues in education, research, funding
  and leadership

\item actively participate in the departmental steering group

\item continue accepting commissions of trust (such as referee for
  papers and grants, examination committees, opponent, etc.)

\item collaborate across traditional subject borders both locally and
  globally (GSS again)
\end{itemize}




\renewcommand{\bibfont}{\small}
\bibliographystyle{unsrtnat}
%\bibliography{../../bibtex/genprog,../../bibtex/jp.short,../../bibtex/misc}
\bibliography{PatrikJanssonProfLect}

\end{document}

\appendix
\newpage
\section{Contributions to publications}
\bibliographystyle{abbrvnat}
\nobibliography{../../bibtex/genprog,../../bibtex/jp.short,../../bibtex/misc}

%\providecommand{\citeentry}[1]{\item \bibentry{#1}.}
\providecommand{\citeentry}[1]{\item \bibentry{#1}.

}

Below is a short description of my contribution to the publications.

\subsection{Journal articles}
\begin{itemize}
\citeentry{janssonjeuring-polyunify}
%
I wrote this paper based on results from my MSc thesis.

\citeentry{janssonjeuring-dataconv}
%
This paper was mainly my work, based on my PhD thesis in 2000 (but
journal publication took some time).

\citeentry{benkedybjerjansson2003:gendepty}
%
I am responsible for the majority of the technical contributions of
this paper, and for most of the implementation.

%  \citeentry{danielssonetal06:fastandloose:SIGPLANNotices}
%  \citeentry{Rodriguezetal2008:CompGenLib:SIGPLANNotices}
\citeentry{MuKoJansson2009AoPA}
%
I initiated this work and wrote the conference paper version together
with Mu, while Ko (supervised by Mu) did most of the implementation.
%
My part is the generic Agda library, while Mu is responsible for the
optimisation applications and the extensions to a journal version.

\citeentry{bernardy_generic_2010}
%
I initiated the work on this paper which is the final result of over a
year of meetings between me, Bernardy, Zalewski and Schupp in the
Software Methodologies and Systems group.
%
I worked out the common terminology, most of taxonomy and quite a bit
of writing (and re-writing).
%
I pulled at the journal version together with Bernardy.

\citeentry{bernardy_proofs_2012}
%
The main contributions of this paper are shared between all three
authors.

\end{itemize}

\subsection{Editor of refereed proceedings}
\begin{itemize}
\citeentry{wgp09proceedings}
%
I was the chair of the Workshop on Generic Programming, with Schupp as
co-chair.
%
I led the programme committee and edited the proceedings.

\end{itemize}
\subsection{Articles in refereed proceedings}
\begin{itemize}
\citeentry{janssonjeuring1997a}
%
This paper was based on my implementation of the language extension
PolyP and the theory I developed together with Jeuring.

\citeentry{janssonjeuring-polylib}
This paper was mainly my work.

\citeentry{janssonjeuringdc}
This paper was part of my PhD thesis.
%
An improved and extended journal version was published as
\cite{janssonjeuring-dataconv} (see above).

\citeentry{janssonjeuringWGP00:rewriting}
%
This workshop paper was also part of my PhD thesis.

\citeentry{norelljansson2003:PolyHaskell}
%
I initiated and led the project and also did a fair bit of writing and
a lot of editing.

\citeentry{danielssonjansson04:chasingbottoms}
%
I pulled at this project to gain a deeper understanding of how domain
theory can be used to reason in detail about Haskell programs.

\citeentry{norelljansson04:ProtoGen}
%
I wrote parts of this paper and edited it several times.

\citeentry{danielssonetal06:fastandloose}
%
I pulled at this project after initial ideas by Gibbons and Hughes.
%
Danielsson did most of the ground work.

\citeentry{janssonjeuring2007:Testing}
%
Based on some initial implementation work by a group of students in
Jeuring's Generic Programming seminar, I improved it to a working
implementation and an accepted paper.

\citeentry{mukojansson08:mpc:dcc}
%
I initiated this work and wrote the paper together with Mu, while Ko
(supervised by Mu) did most of the implementation.
%
My part is the generic Agda library, while Mu is responsible for the
optimisation applications.

\citeentry{SMS-Bernardy:2008:CCC}
%
This paper was the first result of a series of meetings between me,
Bernardy, Zalewski, Schupp and Priesnitz in the Software Methodologies
and Systems group, trying to understand ``Generic Programming'' in
depth.
%
I did quite a bit of writing and (re-)structuring (this paper saw
several versions before it reached its final state).

\citeentry{Rodriguezetal2008:CompGenLib}
%
I did a lot of implementation work (the common test suite and
many of the instances), writing and coordination between the group of
authors of other generic libraries.

\citeentry{LinckeJanssonetalDSLWC2009}
%
The Haskell part is based on Ionescu's PhD thesis and extended by me,
the \Cpp{} part shared between Lincke and Zalewski, while I pulled at the
project and wrote much of the paper and edited all of it in several
iterations.

\citeentry{bernardy_testing_2010}
%
I initiated and supervised the project, and I contributed with the
idea to use embedding-projection-pairs to obtain a more general
theory.

\citeentry{bernardy_parametricity_2010}
%
The main contributions of this paper are shared between all three
authors.

\citeentry{BNFC-meta-Haskell2011}
%
The contributions were shared but Duregård did most of the
implementation work.

\citeentry{jeuringHaskell12ClassLaws}
%
The main work on the paper was done by Jansson and Jeuring.

\citeentry{duregardHaskell12Feat}
%
The contributions and the writing were shared between the authors with
Duregård responsible for most of the implementation.

\citeentry{ionescujansson:LIPIcs:2013:3899}
%
Ionescu did most of the work but we had many discussions and editing
sessions to get to the final paper and the associated proof code.

\citeentry{ionescujansson2013DTPinSciComp}
%
Ionescu did most of the work but we had many discussions and editing
sessions to get to the final paper.

\end{itemize}


\subsection{Pedagogical papers}
\begin{itemize}
\citeentry{niklasson08:IMPACT}
%
The underlying project was led by the three authors, while the paper
was written by Niklasson and Jansson.

\citeentry{niklasson09:IMPACT_Eval}
%
This paper is mainly due to Niklasson.

\citeentry{niklasson09:IMPACT_SEFI}
%
I did the majority of the work on this paper.

\citeentry{Jansson_QA_IMPACT_2010}
%
The indicated chapters describe my own work and are written by me.

\end{itemize}
\subsection{Invited tutorials}
These tutorials (published in Springer's Lecture Notes in Computer
Science) are among my five most cited publications.

\begin{itemize}
\citeentry{jeuringjansson-afp}
%
These lecture notes were written together with Jeuring, but I did most
of the implementation.

\citeentry{backhouseetal98}
%
Initially started by Backhouse and Meertens, these lecture notes were
merged with material by me and Jeuring and grew into almost 90 pages.
%
I was the main responsible for around a third of the text, and for the
underlying implementation.

\end{itemize}
\subsection{Theses: PhD, Licentiate, MSc}
\begin{itemize}
\citeentry{jansson95}
\citeentry{jansson97a}
%
These theses are mainly my own work.

\citeentry{jansson-phdthesis}
%
The PhD thesis is a monograph based on significantly reworked versions
of earlier papers.

\end{itemize}
\subsection{Publicly available implementations}
\begin{itemize}
\citeentry{jansson:PolyPcompiler}
I implemented this compiler 1997--2001.

\citeentry{jansson:PolyP2compiler}
We implemented this version together, 2002--2004.

\citeentry{jansson00:polywww}

\item
I have also participated in the dev. of the advanced computer aided
reasoning engine Agda;

U. Norell et al. \emph{Agda --- a dependently typed programming language and
theorem prover.} Available from the Agda wiki
\url{http://wiki.portal.chalmers.se/agda}, 2009.

\item From 2015 I have been very active on GitHub
  (\url{https://github.com/patrikja}) with daily contributions of open
  source software or text.

\end{itemize}

%%% Local Variables:
%%% mode: latex
%%% TeX-master: t
%%% End:
